\documentclass[english,t,12pt,pdftex,mathserif]{beamer}

\usepackage[utf8]{inputenc}
\usepackage[T1]{fontenc}
\usepackage{babel}
\usepackage{helvet}
\usepackage{times}
\usepackage{courier}
\usepackage{amsmath}
\usepackage{mathptmx}
\usepackage{graphicx}
\usepackage{color}

\hypersetup{pdfstartview={FitH}}

\usetheme{default}
\usecolortheme{beaver}

\newcommand\kuva[2]{\includegraphics[scale=#1]{#2}}

\newcommand\kuvasivu[4]{
  \begin{minipage}[t]{#1\linewidth}
    #4
  \end{minipage}
  \raisebox{-\height}[0pt][0pt]{
    \includegraphics[width=#2\linewidth]{#3}
  }
}

\newcommand\blue[1]{%
  \textcolor{blue}{#1}%
}

\newcommand\red[1]{%
  \textcolor{red}{#1}%
}

\newcommand\green[1]{%
  \textcolor{green}{#1}%
}

\title{Speeding up generalized edit distance calculation using
  Aho-Corasick automaton}

\author{Jani Rahkola}
\institute{University of Helsinki, Department of Computer Science}
\date{\today}

\begin{document}

\selectlanguage{english}

\frame{\titlepage}

\frame{
  \frametitle{Generalized edit distance (g.e.d.)}
  \framesubtitle{}

  \begin{itemize}
  \item The minimal cost of turning string $A$ to string $B$ using
    predefined rules

  \item A rule is a 3-tuple $(\alpha, \beta, c)$ which states that
    turning substring $\alpha$ to substring $\beta$ has a cost of c

  \item In my implementation leaving be symbols that already match has
    a cost of 0
  \end{itemize}
}

\frame{
  \frametitle{Algorithm}

  \begin{itemize}
  \item The dynamic programming algorithm known for calculating en
    edit distance with unit costs can be generalized for the
    generalized edit distance

  \item We keep a matrix of $|A|+1 \times |B|+1$ and calculate the
    g.e.d. for every $(i, j)$

  \item That is, the value in $(i, j)$ will be the g.e.d. between
    substrings $A[:i]$ and $B[:j]$ (with exclusive indexing)

  \item The evaluation can be done in row or column wise order

  \item But my implementation actually stores just promises of
    computations in the matrix
  \end{itemize}
}

\frame{
  \frametitle{g.e.d. $(i, j)$}

  \begin{itemize}
  \item For every rule that $\alpha$ is a suffix of $A[:i]$ and
    $\beta$ is a suffix of $B[:j]$ we add the cost of the rule to the
    cost of turning the remaining prefix of $A[:i]$ to the remaining
    prefix of $B[:j]$

  \item The minimum of the sums is the g.e.d. $(i, j)$

  \item The difference between the basic and Aho-Corasick augmented
    algorithms is the way that they use to determine the applicable
    set of rules for every $(i, j)$
  \end{itemize}
}

\frame{
  \frametitle{The basic algorithm}

  \begin{itemize}
  \item Runs in $O(nm||rules||)$ time where
    $||ruless|| = \sum_{r \in rules} (|r.\alpha| + |r.\beta|)$

  \item At every $(i, j)$ the algorithm goes over the whole set of
    rules and does a naive check whether $\alpha$ is a suffix of
    $A[:i]$ and $\beta$ is a suffix of $B[:j]$
  \end{itemize}
}

\frame[shrink]{
  \frametitle{The Aho-Corasick augmented algorithm}
  \begin{itemize}
  \item Runs in $O(nm|rules|)$ time

  \item We build two Aho-Corasick automatons, one for every $\alpha$
    and one for every $\beta$ in rules

  \item Every state of the automaton has a set of indexes to those
    rules whose $\alpha$ (or $\beta$) is the suffix of the string red
    by the automaton

  \item We make sure that when evaluating g.e.d. $(i, j)$ we have
    pushed $A[:i]$ and $B[:j]$ to the respective automatons

  \item Then the applicable set of rules is the intersection of the
    automaton states' outputs

  \item The output sets are implemented as Java BitSet which has an
    $O(n/w)$ intersection

  \item The construction of the automatons adds an $O(||rules||)$ to
    the overall time, but has to be done only once for a constant rule
    set
  \end{itemize}
}

\frame[plain]{
  \frametitle{Benchmarking}
  Matching a DNA pattern to a reference while allowing non-overlapping inversions
  \kuva{.5}{chart_1.png}
}

\frame[plain]{
  \frametitle{Benchmarking}
  A linearly growing rule that matches at every position
  \kuva{.5}{chart_2.png}
}

\frame[plain]{
  \frametitle{Benchmarking}
  A linearly growing rule in which $\beta$ has a mismatch at the last
  symbol checked by the naive suffix check
  \kuva{.5}{chart_3.png}
}

\frame{
  \frametitle{Applications}
  \begin{itemize}
  \item DNA inversion problem has faster algorithms, but this approach
    allows for weighting the different inversions

  \item Any other problem of edit distance, that needs a huge set of
    weighted rules, maybe given by a machine learning system

  \item Generalized edit distance has been used as an alternative
    for traditional spell checkers
  \end{itemize}
}

\end{document}
